\section{Ensembles}
Ensembles sind Modelle, die bekannte Classifier miteinander
kombinieren. Damit soll ein Overfitting vermieden werden
und die Vorhersagen sollen durch leichte Variationen verbessert werden.

\subsection{Bagging}
Bagging basiert darauf, dass man Randomisierung benutzt,
um die erstellten Modelle unseres Lern-Algorithmus (z.B. 
zur Erstellung eines Entscheidungsbaumes) leicht zu variieren,
um damit "'lokale Schwächen"' der einzelnen Modelle auszugleichen.
So kann Overfitting leicht verhindert werden.

Die einfachste Variante der Randomisierung ist eine Art von Sampling.
Gegeben unseren Trainingsdatenbestand. Aus diesem ziehen wir nun eine
bestimmte Anzahl an neuen Datenbeständen \textbf{mit} Zurücklegen.
Dadurch können einzelne Elemente auch mehrfach oder andere auch gar
nicht vorkommen. Wir erwarten nun, dass die Modelle, die jeweils auf einem
anderen Datenbestand erstellt werden, leicht unterschiedlichen sind. Am besten
wählen wir hier einen Lern-Algorithmus, der nicht stabil ist, also auch schon für
relativ kleine Veränderungen im Datenbestand unterschiedliche Modelle liefert.

Das Ergebnis unseres Ensembles ist dann per Voting im kategorischen
oder als Durchschnitt der Ergebnisse der einzelnen Classifier
im numerischen Fall zu bestimmen. Bagging ist auch eine gute Möglichkeit,
die Güte von Wahrscheinlichkeitsvorhersagen im Kontext von Entscheidungsbäumen
zu verbessern. Das naive Bestimmen der relativen Häufigkeiten ist nicht
unbedingt statistisch signifikant. Aber durch die unterschiedlichen Ergebnisse
im Ensemble gleichen sich die Fehler der einzelnen Bäume aus, wodurch die
Wahrscheinlichkeiten wieder brauchbar werden.

Wenn wir nicht nur die Wahrscheinlichkeiten wollen, sondern uns eher
die Kosten interessieren, dann können wir wie bereits in früheren Abschnitten
die Conditional Risk Function
\[ R(i \mid x) = \sum_j P[j \mid x] \cdot Cost(i,j)\]
benutzen. Hierbei ist \(i\) die vorhergesagte Klasse und \(j\) die tatsächliche
Klasse. Betrachten wir unseren Trainingsdatenbestand: Es ist nun egal,
ob wir die echte Klassenzugehörigkeit explizit gegeben haben, was wir brauchen
ist die Wahrscheinlichkeit für jedes Objekt zu einer bestimmten Klasse zu gehören.
Dann lässt sich damit das Conditional Risk berechnen und wir wählen als "'echte"'
Klasse nun diejenige, die die geringsten Kosten verursacht. Diesen Vorgang nennt
man \textbf{Relabeling}. Indem wir jetzt die Kosten "'implizit"' in den 
Trainingsdaten versteckt haben, können wir einen beliebigen Classifier nehmen
und ihn auf unseren Daten lernen lassen mit dem Ergebnis, dass dieser durch das
Relabeling automatisch die Kosten berücksichtigt. Bei Bagging nennt man dies
auch \textbf{MetaCost}.

Wie bereits erwähnt gibt es verschiedene Arten der Randomisierung. Neben
Sampling gibt es solche, die sich mehr auf die lernenden Verfahren beziehen,
z.B. bei Entscheidungsbäumen wird nun zufällig aus den N-besten Splits einer
gewählt, oder bei NN-Classifiern wird die Distanzberechnung nur in bestimmten,
unterschiedlichen zufälligen Teilräumen (also nach Anwendung von PCA oder SVD) 
ausgeführt.

Bagging kann auch Wahrscheinlichkeiten als Vorhersage generieren: Geben die
einzelnen Classifier bereits Wahrscheinlichkeiten aus, so kann man einfach den 
Mittelwert nehmen. Werden absolute Klassenvorhersagen getroffen, so kann 
das Ensemble einfach die relativen Häufigkeiten ausgeben.

\subsection{Boosting}
Boosting ist eine andere Variante von Ensembles, die darauf abzielt, den Fokus
der Classifier auf Datenobjekte zu lenken, die oft falsch vorhergesagt werden.
Während bei Bagging die Modellgenerierung unabhängig voneinander abläuft,
bauen die Modelle beim Boosting aufeinander auf.

Zunächst werden allen Datenobjekten gleiche Gewichte zugewiesen. Dann berechnet
lernt man seinen Classifier auf den Daten und berechnet den Fehler. Ist der Fehler
0 oder größer gleich \(0.5\), dann wird das Modell verworfen. Ein perfekter Classifier
wird verworfen, da unsere Berechnungen später sonst nicht mehr funktionieren.
Ein Fehler \(\geq 0.5\) zeigt an, dass unser Classifier nicht besser ist,
als naives Raten. Wenn wir also einen zulässigen Fehlerwert haben,
wir das Gewicht jedes korrekt klassifizierten Datenobjektes verringert,
indem es mit \(\frac{e}{1-e}\) multipliziert wird, wobei \(e\) der Fehler ist.
Dadurch wird bei geringen Fehlerraten das Gewicht der richtig klassifizierten 
Objekte sehr klein. Ein Normalisierung aller Instanzgewichte führt zu einer
entsprechend größeren Gewichtung der falsch geschätzten Instanzen. Mit diesen
Gewichten wir nun ein neues Modell gelernt.

Aber wie kann man nun die Gewichtung in den Lern-Algorithmus "'drücken"'?
Prinzipiell ist das mit jedem Classifier ganz einfach möglich: Man wendet 
Resampling wie bei Bagging an, wobei nun die Gewichte den Wahrscheinlichkeiten
entsprechen, mit denen ein Objekt gezogen wird. Dadurch sind die Objekte mit 
höherem Gewicht im Datenbestand überrepräsentiert und unser Lern-Algorithmus
fokusiert sich auf diese besonders stark. 

Hat man alle seine Classifier beisammen, so kann man auch noch eine 
Gewichtung der Vorhersagen der einzelnen Classifier vornehmen. Hierfür
wird ihre Confidence \(conf = -\log \frac{e}{1-e}\) berechnet, die nun
als Gewicht für die Vorhersage dient. Boosting gibt als Ergebnis dann die Klasse
zurück, deren Gewicht am größten ist. Hier gilt, dass für kleine Fehlerraten der 
Logarithmus schnell groß wird, wodurch gute Classifier ihrer Vorhersage viel
Gewicht verleihen können.

Es gibt auch eine Variante von Boosting, die mit einem Greedy-Prinzip versucht,
nur dann neue Classifier zum Ensemble hinzuzufügen, wenn deren Fehler geringer
ist, als der des Vorgängers. Hier ist jedoch zum einen die Komplexität der Modelle
zu bemängeln und zum anderen führt ein solches Verfahren schnell zu Overfitting.

\subsection{Stacking}
Der Unterschied zwischen Stacking und den anderen beiden Ensembles ist,
dass man sich bei Stacking nicht nur auf eine einzige Art von Classifier festlegt,
sondern die Möglichkeit hat, mehrere miteinander zu kombinieren.
Dies funktioniert folgendermaßen: Auf \textbf{Level 0} nimmt man beliebige
herkömmliche Classifier (z.B. Naive Bayes, Decision Tree, Instanz-basierten Lerner)
und auf \textbf{Level 1} benutzt man ein \textbf{Meta-Modell}, welches aus
den Ergebnissen aus Level 0 eine Aussage ableitet. Im Prinzip muss das Meta-Modell
noch nichtmal großartig komplex sein; die meiste Arbeit wurde bereits auf 
Level 0 erledigt. Man muss jedoch aufpassen, dass die Aussagen der Classifier 
richtig gewichtet werden. 

Die Gefahr beim Stacking ist, dass, wenn man den ganzen Datenbestand für
das Training der Level 0 Lernern nutzt, man unterschiedlich gute Classifier erhält.
Fängt man dann an, seinen Level 1 Lernen auf die bereits verwendeten Daten
anzusetzen, so wird dieser feststellen, dass manche Classifier bessere Vorhersagen
auf dem gegebenen Datenbestand liefern, als andere. Daraus könnte sich für
das Meta-Modell leicht die Regel ableiten lassen, dass es nur den besseren 
Classifiern vertraut und die Inputs der anderen verwirft. 

Deswegen ist es nötig, mit den gegebenen Daten klug umzugehen. Eine einfache
Möglichkeit ist es, Holdout Daten zu verwenden, um dann am Ende unser Meta-Modell
auf "'frischen"' Daten zu trainieren. Eine andere Möglichkeit wäre die 
Anwendung von cross-validation, wodurch dann auch der volle Datenbestand
für das Lernen des Meta-Modells zur Verfügung steht.

\subsection{Interpretation von Ensembles}
Noch ein letzter kurzer Einlass über die Interpretation von Ensembles.
Im Allgemeinen sind Ensembles aufgrund ihrer Struktur und insbesondere bei
Stacking aufgrund ihrer Diversität schwer zu durchblicken. Für den User
ist das Ensemble also sowas wie eine Black-Box, die er nicht verstehen kann.
Man kann dies in gewisser Weise umgehen, indem man das Prinzip von 
Relabeling nutzt: Wir werden unser Ensemble auf unseren Datenbestand an
und relabeln die Daten mit den Vorhersagen des Ensembles. Auf den gerelabelten
Daten wenden wir dann einen einfach Lern-Algorithmus an, z.B. einen
Entscheidungsbaum, der dann implizit das Ensemble "'verinnerlicht"' hat. Nun
hängen die Entscheidungen nur noch von einem Classifier ab und die
Klassifikation / Vorhersage wird transparenter.