\section{Association Rules - Erweiterungen und Anwendungen}
Im Folgenden werden ein paar der Weiterentwicklungen von Apriori besprochen,
die bereits kurz nach dessen Veröffentlichung erschienen sind. Sie nehmen sich
verschiedenen Schwächen von Apriori an und verbessern diese signifikant.

\subsection{Direct Hashing and Pruning}
Es konnte gezeigt werden, dass Apriori vor allem für kleine \(k\) einen
hohen Rechenaufwand benötigt, da in diesem Schritt viele (unnötige) Candidate
Itemsets erzeugt werden. \textbf{DHP} ermöglicht es, diesen \textit{bottleneck}
zu umgehen, indem zum einen für kleine \(k\) die Candidate Itemsets verringert
werden, und zum anderen in den weiteren Schritten die Datenbank verkleinert wird.

\subsubsection{Candidate Itemset Reduktion}
Oft ist es der Fall, dass man viele \textit{large 1-itemsets} hat, also \(|L_1|\)
sehr groß ist. Für die Menge der daraus resultierenden Candidate Itemsets ergibt sich
\(|C_2| = {{|L_1|} \choose 2}\). Es wäre also gut, wenn man schon in frühen Iterationen
die Größe von \(C_2\) deutlich verringern könnte. Dies ist in der Tat möglich,
indem man bereits beim Support Counting im \(k\)-ten Schritt Informationen
über die möglichen \textit{(k+1)-Candidate Itemsets} sammelt. Hierfür werden
alle möglichen \textit{(k+1)}-Teilmengen einer Transaktion \(t\) in eine 
Hashtabelle \(H_{k+1}\) gehasht, wobei in jedem Bucket die Anzahl der darauf 
gehashten Teilmengen
gespeichert wird. Bei der Generierung von \(C_{k+1}\) wird nun mit Hilfe der zuvor 
erstellten Hashtabelle \(H_{k+1}\) grob geprüft, ob der Support für ein \(c\in C_{k+1}\)
überhaupt gegeben ist. Auf diese Weise können bereits viele Candidate Itemsets
ausgeschlossen werden, bevor für diese der Support anhand der Datenbank gezählt
werden muss.

Laut Vorlesung sollte unser Hash-Filter derart gewählt werden, dass unser Arbeitsspeicher
möglichst voll ausgenutzt wird, da er unsere Hauptlimitierung darstellt. Für die
übrigen Vergleiche mit den Transaktionen muss nämlich sowieso auf die Datenbank
zugegriffen werden.

\subsubsection{Datenbankreduktion}
Eine weitere Eigenschaft von DHP ist, dass er schrittweise auch die Tranktionen
entfernt, die nicht mehr groß genug sind, um ein \textit{(large k+1)-Itemset}
im Schritt \(k+1\) zu erzeugen, und unnötige Items aus den Transaktionen entfernt.

Die Reduktion der Items basiert auf der Erkenntnis, dass jede Teilmenge einer 
\textit{large Itemsets} ebenfalls \textit{large} ist. Jedes \textit{item} eines
\textit{large (k+1)-Itemsets} ist genau \(k\)-mal in den \((k+1)\) \textit{k}-Teilmengen
enthalten. Umfasst eine Transaktion einige \textit{large (k+1)-itemsets}, dann ist jedes
Item der \textit{(k+1)-Itemsets} mindestens \(k\)-mal im Candidate Itemset \(C_k\)
vorhanden. Wenn ein Item der Transaktion \(t\) nicht \(k\)-mal in den in \(t\)-enthaltenen
Candidate \(k\)-Itemsets vorkommt, kann es gelöscht werden. Auf diese Weise
verringert sich die Transaktionslänge \citep{Petersohn05}.

\subsubsection{DHP Algorithmus}
DHP geht zurück auf \citet{Park95}. Der dort vorgestellte Algorithmus besteht aus
drei Teilen.

Im ersten Teil werden die \textit{Candidate 1-Itemsets} in einen Hashtree
gehasht, wobei gleichzeitig der Support für die einzelnen Items gezählt wird,
und transaktionsweise alle \textit{2-item subsets} in \(H_2\) gehasht werden. Danach
wird \(L_1\) anhand des gezählten Supports bestimmt.\footnote{Soweit eigentlich
alles fast so wie bei Apriori.}

Im zweiten Teil wird ber der Generierung von \(C_2\) auf \(H_2\) zur Vermeidung
unnötiger \textit{Candidate 2-Itemsets} zurückgegriffen. Hat man seine neuen 
Kandidaten erstellt, so wird transaktionsweise deren Support gezählt, wobei schon
hier diejenigen Items geprunt werden, die nicht in wenigstens \(k\) der Kandidaten
vorkommen (also auch nicht für spätere \textit{large Itemsets} relevant sind). Sollte
die aktuell betrachtete Transaktion nach diesem ersten Pruning noch mehr als \(k\)
Items besitzen, so werden alle \textit{(k+1)-Teilmengen} in \(H_{k+1}\) gehasht,
sofern alle \textit{k-Teilmengen} der \textit{(k+1)-Teilmengen} den nötigen Support
mitbringen, also \(H_k[h_k(y)] \geq s\) für alle \textit{k-Teilmengen y} einer
\textit{(k+1)-Teilmenge} gilt. Damit haben wir auch schon die Hashtabelle für die
Candidate Itemset Generierung der nächsten Iteration erstellt. Wird bei der
Erstellung der Hashtabelle festgestellt, dass ein Item in keinem der 
\textit{(k+1)-Itemsets} enthalten ist, kann es geprunt werden.

Der dritte Teil funktioniert im Prinzip wie der zweite. Da jedoch hier bereits
nur noch mit kleinen \(C_k\) und \(L_k\) gearbeitet wird, ist ein weiteres Filtern
unnötig. Deswegen wird hier die Arbeit mit einer Hashtabelle weggelassen. Man geht
dann in Phase 3 über, wenn es in \(H_k\) nicht mehr genügend Buckets gibt, für die
\(H_k[x] \geq s\) gilt.

Für den genauen Algorithmus und die Auswertung der Ergebnisse, die mit DHP erzielt
werden konnten sei auf das zugrunde liegende Paper verwiesen.

\subsection{Sampling}
Ein Nachteil vieler Mining Algorithmen für Association Rules ist die Menge an
Datenbank Zugriffen, die bei der Ausführung entstehen. Algorithmen, die auf 
Apriori beruhen, arbeiten \textit{level-weise}, es werden also für jede Größe
an Frequent Itemsets erst deren Kandidaten erzeugt, die dann mit der Datenbank
geprüft werden. Gibt es also Candidate Itemsets der Größe \textit{(k+1)}, so 
sind \textit{(k+1)} "'\textit{Database Passes}"' notwendig.

Es wäre effizienter, wenn man die Anzahl der Database Passes reduzieren könnte.
Eine Möglichkeit hierfür ist es, eine viel kleinere Teilmenge (\textbf{Sample})
der Datenbank zu betrachten und anhand des Samples Association Rules für die
ganze Datenbank zu generieren. Tatsächlich ist dies sogar so gut möglich, dass
die wirklich vorhandenen Frequent Itemsets nicht nur approximiert, sondern genau
bestimmt werden können, und das in nicht mehr als \textit{worst case} 2 Database
Passes.
Der Vorteil beim Sampling ist, dass dieses in den Arbeitsspeicher passt, sodass
eine schnelle Abarbeitung möglich ist.

\subsubsection{Negative Border}
Um unsere Samples auszuwerten, brauchen wir zunächst das Konzept der \textbf{Negative
Border}. Nehmen wir an, wir haben eine Menge an Items \(R = \{I_1,\ \dots\,\ I_m\}\),
mit \(I_i \in \{0,\ 1\}\). Hierbei bedeutet \(I_i = 0\ (1)\), dass das Item \(I_i\)
in einer Transaktion nicht enthalten (bzw. enthalten) ist. Diese Transaktionen seien
durch eine Relation \(r\) dargestellt, in der jede Transaktion in einer Zeile steht.
Dann sei \(S\subseteq
\mathcal{P}(R)\) eine Menge von Teilmengen, die hinsichtlich der Inklusion 
abgeschlossen ist.\footnote{Damit ist gemeint, dass alle Teilmengen der in \(S\)
enthaltenen Teilmengen ebenfalls in \(S\) enthalten sind.}
Die Negative Border \(\mathcal{B}d^{-}(S)\) beinhaltet nun alle \textbf{minimalen} Itemsets
\(X\subseteq R\), die nicht in \(S\) liegen.

Für die Menge aller Frequent Itemsets \(\mathcal{F}(r,\ minsup)\) gilt insbesondere
die Abgeschlossenheit bezüglich der Inklusion. Damit enthält \(\mathcal{B}d^{-}
(\mathcal{F}(r,\ minsup))\) alle minimalen Itemsets, die nicht mehr frequent sind.
Für ein anschauliches Beispiel sei auf \citet{Toi96} verwiesen.

Gegeben einem Sample \(s\) unserer Datenbank. Wir bestimmen anhand dieses Samples dessen
Frequent Itemsets \(S:=\mathcal{F}(s,\ lowsup)\) mit Mindestsupport \textit{lowsup}.
Dieser Wert sollte kleiner sein als unser eigentlich gesuchter \textit{minsup}, damit
die im Sample gefundenen Frequent Itemsets mit großer Wahrscheinlichkeit die
eigentlichen Frequent Itemsets unserer Datenbank umfassen. Ist \(S\) bestimmt,
müssen die gefundenen Frequent Itemsets anhand der Datenbank verifiziert werden.
Hier kommt unsere Negative Border ins Spiel. Für die Verifizierung berechnen wir
\[\mathcal{F}:=\ \{X\in S\cup \mathcal{B}d^{-}(S) \mid sup(X,\ r) \geq\ minsup\}.\]
Sind alle gefundenen Frequent Itemsets in \(S\) enthalten, also \(\mathcal{F}(r,\ 
minsup) \in S\), dann gibt es kein Problem und mit unserem Sample haben wir alle
Frequent Itemsets gefunden.

Es kann jedoch auch der Fall auftreten, dass es ein Frequent Itemset \(Y\) gibt,
dass in \(\mathcal{B}d^{-}(S)\) liegt. Ein solches Itemset nennen wir \textbf{miss}.
An und für sich sind misses kein Problem, da sie ja in obiger Betrachtung berücksichtigt
wurden. Das Problem entsteht dadurch, dass es Obermengen von \(Y\) geben könnte, die 
ebenfalls frequent sind. Diese sind jedoch nicht mehr in \(S\cup \mathcal{B}d^{-}(S)\)
enthalten, wodurch sie überhaupt nicht entdeckt werden. Diese nicht entdeckten 
Frequent Itemsets nennen wir \textbf{failures}. Stellt man fest, dass man einen
miss hat, geht man wie folgt vor: Ausgehend von \(S\) wird dieses solange um seine
Negative Border erweitert, bis \(S\) nicht mehr weiter wächst. Mit diesem neuen \(S\)
sind sicherlich alle Frequent Itemsets der eigentlichen Datenbank abgedeckt; es sind
potentiell jedoch auch Itemsets in \(S\) enthalten, die eigentlich keine Frequent
Itemsets sind. Dies ist dem geringeren Schwellenwert geschuldet. Um diese nun zu 
entfernen berechnen wir 
\[\mathcal{F}:=\ \{X\in S \mid sup(X,\ r)\geq minsup\}.\]
Insgesamt wurde die Datenbank also zweimal durchlaufen; dies ist jedoch nur im
unwahrscheinlichen worst case der Fall. Meistens reicht bereits bei einer vernünftigen
Wahl von Samplesize und \textit{lowsup} ein einziger Durchlauf aus. In \citet{Toi96}
sind die Auswirkungen von verschiedenen Samplesizes und Schwellenwerten angegeben.

\subsection{Optimistische Verfeinerung}
Apriori ist wie bereits festgestellt deswegen langsam, weil alle möglichen Frequent
Itemsets ihrer Größe nach sequenziell erzeugt werden. Dies bedeutet exponentielles
Wachstum.

Eine Verbesserung kann erreicht werden, indem einfach Schritte ausgelassen werden.
So können aus den \textit{large 1-Itemsets} direkt die Kandidaten für \(C_3\) oder
\(C_4\) erzeugt werden. Durch diese größere Schrittweite kann es passieren, dass man
über die Negative Border hinaus Kandidaten erzeugt. Stellt man eine solche Überschreitung
fest, so nutzt man aus, dass Apriori auch für das Komplement nutzbar ist, und geht
"'rückwärts"' levelweise wieder zurück, bis man wieder "'unter"' der Negative Border ist.

Durch ein solches Vorgehen ist es möglich, approximativ lineare Laufzeiten zu erreichen, 
was eine deutliche Verbesserung gegenüber der exponentiellen Anzahl an generierten Kandidaten
von Apriori ist.

\subsection{Frequent Pattern Trees}
Die bisher vorgestellten Verfahren folgen dem so genannte \textit{generate and test
paradigma}. Es wurden zunächst Kandidaten für large Itemsets erzeugt, die dann 
mit der Datenbank getestet wurden. Da dies jedoch schon aufgrund der kombinatorischen
Natur der Vorgehen selbst für mittelgroße Datenmengen bei geringen Thresholds für
großen Aufwand sorgt, kann es leicht passieren, dass der Arbeitsspeicher nicht
mehr für die betrachteten Objekte ausreicht. Deswegen wäre eine kompaktere Darstellung
wünschenswert. In der Tat ist dies mit \textbf{Frequent Patter Trees} möglich.
Aus den \textit{FP-Trees} lassen sich dann die Frequent Itemsets direkt produzieren,
ohne erst Kandidaten erzeugen zu müssen.

\subsubsection{FP-Tree Erzeugung}
Das Erzeugen eines FP-Trees ist relativ günstig, da man gerade einmal \textbf{zwei}
Datenbank-Scans benötigt. In einem ersten Scan wird wie bei Appriori die
\textit{Häufigkeit der individuellen Items} gezählt. Im zweiten Durchlauf wird dann
die Baumstruktur erstellt. Hierfür werden die Items in einer Transaktion entsprechend
der gezählten Häufigkeit sortiert. Hierbei muss sichergestellt werden, dass es in
allen Transaktionen die gleiche Sortierung gibt (also im Falle von gleichen Häufigkeiten
kommt ein weiteres Merkmal zur Sortierung ins Spiel, z.B. die Lexikographische Ordnung).
Hierbei können dann bereits die Items aussortiert werden, die nicht einem gewissen
Minimumsupport genügen. Nach dem Sortieren und Prunen einer Transaktion wird diese
dann in den Baum eingefügt.\footnote{Beachte: Die Wurzel eines FP-Trees bleibt leer!
Das erste Item der ersten Transaktione geht also schon als Child der Wurzel in den Baum ein!}
Beim Einfügen werden in den Knoten des Baumes die Anzahl der Transaktionen, die
diese "'durchlaufen"' gespeichert, d.h. fangen zwei sortierte Transaktionen mit den
selben ersten drei Items an, so werden diese die selben Knoten beim Einfügen durchlaufen,
wobei die entsprechenden Counter in den Knoten inkrementiert werden. Auf diese Weise
können Transaktionen mit gleichen Präfixen effizient komprimiert werden.
Hat man alle Transaktionen auf diese Weise verarbeitet,
so darf man auf eine hohe Kompression um die Wurzel des Baumes herum hoffen.
Jedes Item aus dem FP-Tree zeigt auch auf \textbf{ein} weiteres Item mit gleichem
Namen, sofern noch ein solches vorhanden ist, das nicht schon mit einem anderen
gleichnamigen Item verbunden ist. Dies erleichtert später die Traversierung. Es
entsteht als eine Art Item-Sequenz (das wichtige ist, dass es einen Anfang gibt).

Die zweite Komponente unseres FP-Trees ist eine \textbf{Header-Table}. Diese
Tabelle enthält alle individuellen Items, die dem Minimumsupport entsprechen, sortiert
nach absteigender Häufigkeit. Jedes Element dieser Header-Table zeigt auf das gleichnamige
Item im FP-Tree, welches den Anfang der entsprechenden Item-Sequenz bildet. Man kann
nun also leicht über die Header-Tabelle auf die Item-Sequenzen im Baum zugreifen.

\subsubsection{Frequent Pattern Mining}
Eine wichtige Beobachtung im Rahmen von FP-Trees ist, dass \textit{für jedes Item
\(a_i\) alle möglichen Frequent Patterns gefunden werden können, wenn man der
Item Sequenz, die mit \(a_i\) in der Header-Table beginnt, folgt und
die Pfade zu diesen \(a_i\) betrachtet.} Diese Eigenschaft
wird auch \textbf{node-link property} genannt. Sie erlaubt es, dass mit nur
einer Traversierung des FP-Trees alle möglichen Frequent Patterns, die \(a_i\) enthalten,
gefunden werden können.

Betrachte ein \(a_i\) und den dazugehörigen Pfad zu diesem Item. Dann gilt
für die \textit{counts} der Items im Präfix-Pfad zu \(a_i\), dass diese den selben
Wert haben, wie \(a_i.count\). Haben mehrere Pfade eine Überschneidung am  Anfang,
so ist der count der Items in der Schnittmenge die Summe der counts der jeweiligen Pfade.
Mit anderen Worten: Die counts der Items im durch \(a_i\) bedingten Baumes ergeben 
sich dadurch, dass die Blätter die counts der entsprechenden \(a_i\) erhalten und sich
diese in ihren Elternknoten aufsummieren.

Zusammenfassend können also Frequent Patterns wie folgt gefunden werden.
\begin{enumerate}
  \item Erstelle den FP-Tree aus der Ursprungsdatenbank.
  \item Betrachte das am seltensten vorkommende Item in der Header-Table.
  Dieses sei mit \(a_i\) bezeichnet.
  \item Betrachte den Subtree, der als Pfade die Pfade zu den \(a_i\) besitzt.
  \item Füge in den Eltern der \(a_i\) die count-Werte der \(a_i\) ein und 
  summiere diese nach oben hin auf. Entferne die \(a_i\) aus dem Subtree.
  Dieser Subtree wird auch als \textit{Conditional FP-Tree of \(a_i\)} bezeichnet.
  \item Prüfe nun, ob es Items in diesem Subtree gibt, deren count noch den
  Minimumsupport erfüllt. Diese Items können zusammen mit \(a_i\) noch zu einem
  größeren Frequent Pattern anwachsen. Angenommen es gäbe ein Item \(a_k\),
  welches im Subtree den Minimumsupport erfüllt. Erstelle nun den Conditional
  FP-Tree of \(a_k a_i\) aus dem Conditional FP-Tree of \(a_i\). Es findet also ein
  rekursiver Aufruf statt. Beachte, dass sich diese Rekursion verzweigt stattfinden kann,
  d.h. man muss erst alle Rekursionsteilpfade fertig berechnen, bevor man den ganzen
  Rekursionspfad abschließen kann.
  \item Gibt es keine Items mehr, die im Conditional FP-Tree noch den Minimumssupport
  erfüllen, so breche diesen Rekusionszweig ab. Gibt es keine Rekursionsmöglichkeiten für
  \(a_i\) mehr, so wiederhole Schritt 3--6 für das nächste Item der Header Table.
  Für dieses muss man
  keine Rücksicht mehr auf die \(a_i\) nehmen, da man ja im vorherigen Schritt bereits alle 
  Frequent Patterns erzeugt hat, die \(a_i\) enthalten.
\end{enumerate}

Stellt man fest, dass man einen Conditional FP-Tree of \(a_i\) erhält, der nur aus einem 
einzigen Pfad
besteht (alle Items in diesem Single Path haben den selben count), so lässt sich dieser leicht
minen: Man muss lediglich alle Itemkombinationen dieses Pfades nehmen und diese mit 
\(a_i\) konkatenieren (\(a_i\) steht am Ende). 

Die Grundidee stammt aus \citet{Han00}; aus dieser Arbeit stammen auch die Beispiele
aus der Vorlesung. Für ein weiteres anschauliches Beispiel sei auf Kapitel 6.3 von \citet{Witten11}
verwiesen.
