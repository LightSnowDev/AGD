\section{Pattern Mining unter Constraints}
\subsection{Constrained Association Rules}
Die bisher vorgestellten Verfahren von Pattern Mining
haben das gemeinsame Problem, dass sich nicht fokussierbar
sind. Es werden immer alle Rules gemined, ganz gleich, ob
diese auch wirklich interessant für den User sind. Besser
wäre es, wenn man stattdessen dem User ein Interface gibt,
welches er zur Einschränkung der Suche nutzen kann. Damit
kann neben relevanteren Ergebnissen unter Umständen auch
eine Verkürzung der Laufzeit einhergehen. 

Das Mining bisher hat sich die Konzepte von Support und
Confidence zu nutzen gemacht, um Items und Transaktionen
zu prunen. Dies soll nun durch das Pruning mit Constraints
ersetzt werden. Ein Item oder eine Transaktion wird dann
geprunt, wenn es/sie nicht mehr alle Constraints erfüllt.

Ein naiver Ansatz wäre nun, zuerst mit den bekannten 
Algorithmen alle Association Rules zu generieren und diese
dann in einem \textit{Post-Pruning} hinterher zu filtern.
Diese Variante ist jedoch nicht effektiv, da
die Laufzeit eines solchen Vergehens offensichtlich mehr
Zeit in Anspruch nimmt, als nötig. Es wäre doch viel besser,
wenn man schon während der Association Rule Erzeugung
die irrelevanten Rules prunen könnte. Das ist in der Tat
möglich, dafür müssen die Constraints jedoch bestimmte
Eigenschaften erfüllen, auf die wir gleich eingehen werden.

Zunächst klassifizieren wir jedoch die möglichen Constraints.
Im Wesentlichen lässt sich zwischen \textbf{Data Constraints}
und \textbf{Rule Constraints} unterscheiden.

\textit{Data Constraints} spezifizieren sehr genau, wie das
Mining Ergebnis auszusehen hat, z.B. "'Finde alle Produkte,
die in \textit{Freiburg} im \textit{Dezember 2014} oft
zusammen gekauft wurden."', oder Einschränkungen auf 
bestimmte Attribute, z.B. (region, price, brand, customer).
Das Problem ist, dass hierbei die Mining Ergebnisse nicht
nur besonders stark eingegrenzt werden; es setzt auch voraus,
dass der User ein sehr gutes Verständnis der Daten hat und
schon eine Vorstellung hat, wie das Ergebnis aussehen sollte.

\textit{Rule Constraints} hingegen wirken nicht so
einschränkend. Sie treffen lediglich Aussagen über die
Struktur der zu minenden Regeln, z.B. "'Nur Frequent Itemsets
der Größe 3."' oder "'Finde alle Paare von günstigen Items, 
die den Kauf eines oder mehrerer teurer
Item(s) gefördert haben."'

Im Folgenden werden wir auf zwei Ansätze eingehen, mit
denen Constraint basiertes Mining möglich ist. Der
Unterschied zwischen den beiden Ansätzen besteht in der
zugrunde liegenden Datenstruktur (vermutlich können
die Konzepte trotzdem auch auf andere Situationen 
verallgemeinert werden).

\subsubsection{Meta-Rule Guided Mining}
Beim Meta-Rule Guided Mining gehen wir von einer
relationalen Datenbank mit Schema aus. Für Datenbanken
kennt man im Allgemeinen bereits schon naive Anfrage-
Sprachen wie SQL. Das Problem ist, dass diese zu restriktiv
sind. 

Eine Abwandlung hiervon ist die \textit{Data Mining Query
Language} (DMQL). Angenommen, wir hätten eine Datenbank, die wie folgt aussieht:
\begin{align*}
	student(\text{name, sID, status, major, gpa, birthplace,
	birthdate, adress})\\
	course(\text{cID, title, dept})\\
	grading(\text{sID, cID, instructor, semester, grade}).
\end{align*}
Eine Anfrage in DMQL hätte dann ungefähr die Form:
\begin{align}
	\text{Finde alle Regeln der Form:}\\
	major(s:\text{ student},\ x)\wedge Q(s,\ y)
	\Longrightarrow R(s,\ z)\\
	\text{from } student\\
	\text{where } birthplace = Canada\\
	\text{in relevance to } major,\ gpa,\ status,\ adress.
\end{align}
In dieser Anfrage sind mehrere Arten von Constraints
enthalten, auch einige, die nicht Teil der Vorlesung waren.
Die bisher kennengelernten \textit{Data} und 
\textit{Rule Constraints} kommen in obigem Beispiel
aber auch vor: In Zeile 2 ist die so genannte \textbf{
Meta Rule} zu finden. In ihrer ersten Komponente stellt
\(student\) eine Data Constraint dar. Ebenso ist Zeile 3
eine Data Constraint, da sie angibt, welche Art von Daten 
als Ergebnis zulässig sind. Auch Zeile 4 ist eine Data 
Constraint. Eine Rule Constraint stellt die Tatsache dar, dass
auf der linken Seite der Regel zwei Prädikate stehen müssen.
Prädikate sind Variablen, die mit Attributen aus der 
betrachteten Datenbank instanziiert werden während dem
Mining Prozess. Das Mining Ergebnis könnte dann die Form
\[
major(s,\text{ Art}) \wedge gpa(s, \text{ Excellent})
\Longrightarrow status(s, \text{ Graduate})
\quad (60\%)
\]
haben, also alle Kanadier, die Kunst studieren und einen
exzellenten Notenschnitt haben, bestehen auch ihr Studium,
wobei diese Regel einen Support von 60\(\%\) hat. Ein 
Kapitel über Meta-Rule Guided Mining findet sich auch in
\citet{HanBOOK}; hier werden jedoch ein paar Konzepte
anders als in der Vorlesung definiert. 

\subsubsection{1-var und 2-var Constraints}
Wenn wir es nicht mit einer ganzen relationalen Datenbank
zu tun haben, sondern nur mit einer einfachen Datenbank
aus unserem Warenkorb-Szenario, dann gibt es die Möglichkeit
mit \textbf{1-var} und \textbf{2-var Constraints} zu 
arbeiten. Wie der Name schon sagt wirkt jeweils entweder
eine Seite der Regel (links oder rechts) oder beide Seiten 
einschränkend. 

\begin{enumerate}
\item \textbf{1-var Constraints:} Eine 1-var Constraint kann eine der
folgenden Formen annehmen.
\begin{enumerate}
\item \textit{Class Constraint:} Hat die Form \(S\subset A\), wobei
\(S\) eine Menge und \(A\) ein Attribut ist. Diese Constraints sagen
aus, dass \(S\) eine Menge von Werten aus der Domain von \(A\) ist.
\item \textit{Domain Constraints:} Hat eine der folgenden Formen.
\begin{enumerate}
\item \(S\theta v\), wobei \(S\) eine Menge, \(v\) eine Konstante aus der
Dömane von \(S\) und \(\theta\in\{=,\neq, <,\leq, >\geq\}\) ist.
Sie sagen aus, dass alle Elemente aus \(S\) in der Beziehung \(\theta\)
zu \(v\) stehen.
\item \(v\theta S\), wobei \(v\) und \(S\) wie oben sind und 
\(\theta\in\{\in,\notin\}\) ist. Sie sagen lediglich aus, dass
\(v\) in \(S\) vorkommen muss (oder nicht vorkommen darf).
\item \(V\theta S\) oder \(S\theta V\), wobei \(S\) wie oben ist,
\(V\) eine Menge von Konstanten aus der Domäne von \(S\) und 
\(\theta\in\{\subseteq ,\not\subseteq , \subset ,\not\subset , =,\neq\}\) ist.
\end{enumerate}
\item \textit{Aggegrate Constraint:} Hat die Form \(agg(S)\theta v\),
wobei \(agg\) eine der Aggregatsfunktionen \textit{min, max, sum, count} und
\textit{avg} ist, und \(\theta\in\{=,\neq,<,\leq,>,\geq\}\) ist. Sie sagt aus,
dass das Aggregat der numerischen Werte aus \(S\) in der Beziehung \(\theta\) zu
\(v\) stehen.
\end{enumerate}
\item \textbf{2-var Constraints:} Eine 2-var Constriant kann eine der folgenden
Formen annehmen.
\begin{enumerate}
\item \(S_1\theta S_2\), wobei \(S_i\) eine Menge und \(\theta\in\{\subseteq,
\not\subseteq, \subset,\not\subset\}\) ist.
\item \((S_1\diamond S_2)\theta V\), wobei \(S_1,\ S_2\) Mengen sind, 
\(\diamond\in\{\cap,\cup\}\) und \(\theta\in\{=,\neq ,\subseteq ,\not\subseteq ,\subset ,
\not\subset \}\) ist.
\item \(agg_1(S_1)\theta agg_2(S_2)\), wobei \(\theta\in\{=,\neq ,< ,\leq , > ,\geq\}\)
ist.
\end{enumerate}
\end{enumerate}


Eine Anfrage unter gegebenen Constraints an die Datenbank
unter Zuhilfenahme der eben vorgestellten 1-var und 2-var 
Constraints ist mit der \textit{constrained association
query} (CAQ) möglich. Diese hat die Form \(\{S_1,S_2
\ |\ \mathcal{C}\}\), wobei \(\mathcal{C}\) eine
Menge an Constraints auf \(S_1,\ S_2\) ist. Ein Beispiel
wäre 
\begin{align*}
\{(S_1,S_2) \mid\
&S_1\subset \text{Item }\&\ S_2\subset \text{Item }\ \&\\
&count(S_1) = 1\ \&\ count(S_2) = 1\ \&\\
&freq(S_1)\ \&\ freq(S_2)\}.
\end{align*}
Diese CAQ fragt nach allen Single Item Paaren, die den 
Mindestsupport erfüllen. Für ausgefeiltere CAQs und Beispiele
sei auf \citet{Ng98} verwiesen.

Beim Apriori Algorithmus und seinen Erweiterungen hat
man sich zwei wichtige Eigenschaften von Frequent Itemsets
zu nutze gemacht, nämlich die, dass alle Untermengen eines
Frequent Itemsets ebenfalls frequent sein müssen, und
dass keine Obermenge eines nicht frequent Itemsets frequent
sein kann. Diese Eigenschaften haben sich zur Generierung
von Frequent Itemsets "'tief in den Algorithmus"' reindrücken
lassen, d.h. dass sie schon während den einzelnen Iterationen 
berücksichtigt wurden. Für unsere Constraints wollen wir
nun im Folgenden auch solche Eigenschaften finden, die es
uns erlauben, die Constraints möglichst weit in unseren
Algorithmus reinzudrücken, wodurch sich der Aufwand im
Post-Pruning erheblich verringert oder sogar ganz entfällt.

\subsubsection{Eigenschaften von Constraints}
\begin{table}[!ht]
  \centering
  \caption{Charakterisierung der Anti-Monotonizität und Succinctness von
	1-var Constraints.}
  \label{tab:var_con}
\begin{tabular}{lcc}
	\toprule
	1-var Constraint & Anti-Monoton & Succinct\\
	\midrule
	\(S\theta v\), \(\theta\in\{=,\leq,\geq\}\) & ja & ja\\
	\(v\in S\) & nein & ja\\
	\(S\supseteq V\) & nein & ja\\
	\(S\subseteq V\) & ja & ja\\
	\(S = V\) & teilwese & ja\\
	\midrule
	\(min(S)\leq v\) & nein & ja\\
	\(min(S)\geq v\) & ja & ja\\
	\(min(S) = v\) & teilweise & ja\\
	\midrule
	\(max(S)\leq v\) & ja & ja\\
	\(max(S)\geq v\) & nein & ja\\
	\(max(S) = v\) & teilweise & ja\\
	\midrule
	\(count(S)\leq v\) & ja & schwach\\
	\(count(S)\geq v\) & nein & schwach\\
	\(count(S) = v\) & teilweise & schwach\\
	\midrule
	\(sum(S)\leq v\) & ja & nein\\
	\(sum(S)\geq v\) & nein & nein\\
	\(sum(S) = v\) & teilweise & nein\\
	\midrule
	\(avg(S)\ \theta\ v\), \(\theta\in\{=, \leq, \geq\}\) & nein & nein\\
	\midrule
	(Frequency Constraint) & (ja) & (nein)\\
	\bottomrule
\end{tabular}
	\\
	\vspace{10pt}
	{\raggedright \(^*\) Aus Platzgründen sind hier die Negationen der 
	jeweiligen Relationen nicht aufgelistet.\\}
\end{table}

Die gewünschten Eigenschaften heißen \textbf{Anti-
Monotonizität} und \textbf{Succinctness}.

Erstere lässt sich wie folgt definieren: \textit{Eine 1-var
Constraint C ist anti-monoton genau dann, wenn für alle
Mengen \(S,\ S'\) gilt:}
\[ S' \subseteq S\ \& \text{ S erfüllt C} \Longrightarrow
S'\ \text{erfüllt}\ C. \]
Das Ziel ist es nun, mithilfe der Anti-Monotonizität möglichst
früh diejenigen Itemsets, die das Constraint nicht mehr
erfüllen, zu prunen, da auch deren Obermengen die 
Constraints nicht mehr erfüllen können.

Die Succintcness-Eigenschaft soll es ermöglichen, schon die
Kandidaten nicht zu erzeugen, die die Constraints nicht 
erfüllen würden. Die Definition ist nicht so straight-forward
wie bei der Anti-Monotonizität, der interessierte Student sei
abermals auf \citet{Ng98} verwiesen. Man sollte sich merken,
dass ein Constraint dann succinct ist, wenn man alle
Itemsets, die das Constraint erfüllen, "'kurz und knapp"'
aufschreiben kann. Das heißt, man kann bereits alle möglichen
Frequent Itemsets und Association Rules erzeugen, ohne überhaupt
jemals auf die Datenbank geblickt haben zu müssen. Dies ist z.B.
mit regulären Ausdrücken möglich.

Dies ist bei Weitem noch nicht der Wahrheit letzter Schluss. Es wurden
Effektiv gerade mal die grundlegenden Konzepte erklärt. Die genaue
Implementierung der Constraints in den Algorithmus stellt ein ganz
anderes Problem dar. Es ist leicht zu erkennen, dass aufgrund der
unterschiedlichen Kombinationen von Constraint Eigenschaften, z.B.
anti-monoton aber nicht succinct oder sowohl anti-monoton als auch 
succinct, andere Strategien beim Einfügen in den verwendeten Algorithmus
benutzt werden müssen,
eine Übersicht findet sich in Tabelle~\ref{tab:var_con}. 
Da dies aber über den Stoff der Vorlesung hinaus
geht, beschäftigen wir uns lieber mit einer anderen Variation: Dem Finden
von \textit{Frequent Sequences}.

\FloatBarrier
\subsection{Constrained Frequent Sequence Mining}
Anstelle von Itemsets kann es auch interessant sein, sich Folgen von Items
anzuschauen. Eine Folge von Items ist eine Menge \(S = \langle s_1,\ \dots,\ 
s_n\rangle\), bei der die Reihenfolge fest gegeben ist. Eine Teilfolge \(T\)
lässt sich aus \(S\) generieren, indem beliebig viele \(s_i\in S\) aus \(S\)
für \(i \leq n\) entfernt werden. Aus wirtschaftlicher Sicht kann dies
interessant sein, wenn z.B. einen Einkaufstrend für schwangere Frauen in den
Daten erkennen kann. Hat man so etwas wie ein Pattern für das Einkaufsverhalten
von Frauen über den gesamten Zeitraum der Schwangerschaft, so könnte man diese
Informationen nutzen, um jeweils zugeschnittene Werbung für potentiell Schwangere
zu schalten.

Das Mining nach Frequent Sequences ist dem von Association Rules nicht unähnlich.
Das Grundgerüst stellt wieder der Appriori Algorithmus dar. Dieser wird dadurch
erweitert, dass innerhalb jeder Iteration sowohl Support-basiert, als auch
Constraint-basiert geprunt wird. Constraints sind in diesem Falle meist 
\textit{Reguläre Ausdrücke}, da sie zum einen ein mächtiges Beschreibungswerkzeug
und zum anderen leicht in deterministische endliche Automaten überführbar sind.
Auch für  RegEx gibt es die Konzepte von Anti-Monotonizität und Succinctness, wobei
Succinctness nach dem Verständis des Verfassers schon sowieso für alle RegEx gilt.

Ein Beispiel für eine \textit{nicht} anti-monotone RegEx wäre \((ab)^*\). Betrachtet
man die Folge \(abab\), welche die Constraint erfüllt, so besitzt diese diverse 
Teilfolgen, z.B. \(aab\), die die Constraint nicht mehr erfüllen. Beim Mining
wäre es egal, ob die Folge \(aab\) möglicherweise frequent ist, da sie aufgrund
des Constraint-basierten Prunings (welches vor dem Support-basierten geschieht,
da man sonst das Constraint nicht effektiv genutzt hätte) niemals in \(L_3\)
erscheinen wird. 

Auf den ersten Blick scheint dies ja wünschenswert zu sein.
Man erhält gerade die Frequent Sequences, die das Constraint erfüllen und man
muss sich nicht mit Sequencen herumschlagen, die zwar frequent sind, jedoch das
Constraint nicht erfüllen und damit für den User uninteressant sind. The devil
is in the detail.

Wenn unser Constraint nicht anti-monoton ist, dann kann es passieren, dass eine
Teilfolge existiert, die unser Constraint nicht erfüllt, obwohl die Ausgangsfolge
konform ist. Schaut man in den SPIRIT Algorithmus \citep{Garofalakis99}, so sieht man aber, dass dann
eine solche Folge für die späteren Iterationen nicht mehr betrachtet wird, man hat
für diese Teilfolgen keine Informationen mehr über ihren Support-count. Das hat
zur Folge, dass diese Folgen nicht mehr zum Support Counting von längeren Folgen
benutzt werden können, d.h. die Effektivität unseres Support-basierten Prunings
wird erheblich verringert. 

Dieser Trade-Off zwischen Constraint- und Support-basiertem Pruning hängt in
großem Maße von der Restriktivität unseres gewählten Constraints ab. Ist es sehr
restriktiv (und damit im Allgemeinen auch oft nicht mehr anti-monoton), so bleibt
offensichtlich hinterher "'nicht mehr viel"' für das Support-basierte Pruning übrig.
Es konnte in Studien gezeigt werden, dass ein zu starkes Constraint sich sogar negativ
auf die Laufzeit unseres Algorithmus auswirken kann. Ein Möglicher Lösungsansatz
ist die \textbf{Relaxation} unseres Anfangsconstraints \(C\). Hierbei wird einfach ein
Constraint \(C'\) gewählt, für das gilt:\textit{ Wenn eine Folge \(C\) erfüllt, dann erfüllt
sie auch \(C'\).} Es gibt verschiedene Abstufungen der Stärke der Relaxation, es
reicht aber, sich zu merken, dass \(C'\) "'mehr durchlässt"' als \(C\). Am besten
wählt man \(C'\) derart, dass es anti-monoton ist, sodass man alle Teilfolgen einer
erzeugten zulässigen Folge behalten und damit für das Support-basierte Pruning nutzen
kann. Für genauere Infos und die Ergebnisse der verschiedenen Relaxationsstufen sei
auf \citet{Garofalakis99} verwiesen.
