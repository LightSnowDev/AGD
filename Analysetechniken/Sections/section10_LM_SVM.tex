\section{Lineare Modelle und SVMs}
Die Lineare Regression kennen wir ja noch aus Statistik 2.
Gegeben einen Datenbestand mit numerischen Attributwerten.
Wir wollen nun Gewichte finden, sodass sich ein bestimmtes Attribut
als gewichtete lineare Summe der anderen Attribute beschreiben lässt.
Dies wollen wir für alle unsere Attribute tun.

Bei der Gerade, die wir dabei erhalten, wird die Steigung durch
die Gewichte bestimmt und der Bias (y-Achsenabschnitt) bestimmt
den Offset.  

Wir wollen nun mit unseren Gewichten möglichst die Abweichungen
unserer Vorhersage von den tatsächlichen Daten minimieren, also
\[ \min \sum_{i=1}^{n}\left( x^{(i)} - 
\sum_{j=0}^k w_j \cdot a_j^{(i)}\right)^2 .\]
Dabei sei \(x^{(i)}\) das i-te Objekt.

Im Folgenden wollen wir uns mit Regression im Kontext von
Klassifikation beschäftigen: Ist es möglich, eine Gerade zu finden,
die 2 Klassen von Daten voneinander trennt?

\subsection{Logistic Regression}
Diese Regressionsart funktioniert gut bei binärer Klassifikation.
Das Problem bei binärer Klassifikation mit linearer Regression ist,
dass die Gerade eigentlich auf dem Intervall \([0,1]\) angeben sollte,
für welchen Werte auf der x-Achse welche Wahrscheinlichkeit für
die jeweilige Klasse angenommen wird. Die Gerade ist jedoch nicht auf
dem Intervall \([0,1]\) beschränkt und ab einem bestimmten Wert
für \(x\) liefert uns die Gerade keine Wahrscheinlichkeiten mehr.
Sinnvoller wäre hier eine Art S-Kurve, die jeweils gegen 0 und 1 konvergiert.
Die so genannte logistische Funktion hat diesen Verlauf:
\[P[Klasse\ |\ x]= \frac{1}{1+e^{-w_0 - w_1 x_1 - \dots - w_k x_k}}\]
Hierbei ist \(k\) die Zahl der Attribute.

Diese Funktion hat im Wesentlichen 2 Parameter, nämlich die Stauchung
und der Abschnitt auf der x-Achse. Diese lassen sich nicht notwendiger
Weise direkt auf die Gewichte aufteilen, sie hängen komplexer von 
den Parametern ab. Aber lassen wir das auf sich beruhen.

Durch einfache Umformungen kann man die logistische Funktion in eine
Form bringen, in der  die gewichtete Linearkombination auf einer Seite
alleine steht, also 
\[
w_o + w_1 x_1 + \dots =
\log \left(\frac{P[Klasse \mid x]}{1 - P[Klasse \mid x]}\right).
\]
Das nennt man die Logit Transformation. Hierauf lässt sich wieder die
normale Lineare Regression anwenden. Der Fehler lässt sich ähnlich
dem Bernoulli-Experiment mittels
\[
\sum_{i=1}^n (1-x^{(i)})\log(1-P[K \mid x^{(i)}]) + 
x^{(i)}\log(P[K \mid x^{(i)}])
\]
ermitteln. Hierauf lässt sich ein iteratives Verfahren anwenden, dass die Gewichte
in die richtige Richtung verschiebt. Diese Formel soll maximiert werden.
Es lässt sich auch stattdessen die robustere log-Variante berechnen. 
Generell funktioniert das Verfahren nur dann wirklich gut, wenn die 
Daten sich auch halbwegs gut separieren lassen.

Da der Teil mit den Perzeptoren in der Vorlesung ausgelassen wurde, geht 
es mit den SVMs weiter.

\subsection{Support Vector Machines}
In den meisten Fällen ist es nicht so, dass sich Daten einfach durch
eine Gerade voneinander trennen lassen, z.B. da sie teilweise vermischt 
sind. Wir bräuchten also statdessen ein Polynom, dass unsere Grenze 
beschreibt. Dieses ist im Allgemeinen nicht leicht zu berechnen und hat
eine komplexe Struktur. Die Idee ist es nun, die Daten in einen Raum
mit höherer Dimension zu projizieren, sodass sich dort leichter
eine trennende Ebene finden lässt (z.B. \([a_1,a_2]\) wird zu
\([a_1^3,a_1^2 a_2, a_1 a_2^2, a_2^3]\), also von 2D zu 4D).

Aber zunächst nocheinmal eine grundlegende Definition:
Gegeben zwei separierbaren Datenmengen. Die \textbf{Maximum
Margin Hyperplane} zwischen diesen beiden Mengen ist die Ebene, die
die Daten bei größt möglichem Abstand jeweils voneinander trennt, die 
also "'am mittigsten"' zwischen den beiden Menge verläuft. Diese lässt
sich relativ leicht finden für separierbare Daten: Man berechnet für beide
Menge die konvexe Hülle, findet die kürzeste Strecke zwischen den
beiden und die Mittelsenkrechte dieser Strecke wäre unsere Maximum
Margin Hyperplane (hier im einfachen Fall von 2 Dimensionen). Man sieht,
dass eine solche Gerade (bzw. Ebene) nur von ein paar wenigen Punkten
bestimmt wird, nämlichen den jeweils nächsten Datenpunkten.
Diese heißen \textbf{Support Vectors}. Es können pro Datenmenge
durchaus mehrere Support Vectors existieren, ihre Anzahl ist im 
Allgemeinen trotzdem recht klein. Dadurch, dass die Hyperplane nur von
wenigen Vektoren abhängt, ist dieses Verfahren recht robust, da 
ein Hinzufügen von neuen Datenobjekten nur selten eine Anpassung
der Hyperplane nach sich zieht. Die Maximum Margin Hyperplane lässt
sich auch in Abhängigkeit der Support Vectoren darstellen:
\[
x = b + \sum _{i \in SV} \alpha_i y_i \cdot \langle a(i),a \rangle
\]
Hierbei ist \(b\) eine Konstante, \(i\) ist aus der Menge der Indizes 
der Support Vectors, \(\alpha_i\) sein ein Gewicht, \(y_i\) ist die 
Klasse des i-ten Objektes, \(a(i)\) ist der Vektor des i-ten Support 
Vectors und \(a\) ist der zu klassifizierende Vektor. Normalerweise
würde man die Hyperplane als eine Linearkombination darstellen, jetzt
ist aber der Bezug zu den Support Vectors gegeben. Wer sich fragt,
wie man auf die obige Darstellung kommt... Magie und Dualisierung.

Unser Ziel ist es nun, die Maximum Margin Hyperplane in den verschiedenen
Dimensionen zu finden und diese durch Rücktransformation in unserem
Ursprünglichen Datenraum als Trennlinie für unsere Daten zu nutzen.
So schön die Idee auch ist, das Problem stellt der Rechenaufwand dar:
Dieser wächst nämlich kubisch in der Anzahl der zu berechnenden 
Gewichte, was zu einem großen Teil durch die Berechnung von 
Skalarprodukten kommt.

Dieses Problem lässt sich jedoch umgehen: Mit den so genannten
\textbf{Kernel-Functions} lässt sich die Performance von SVMs
signifikant verbessern, bzw. überhaupt erst ermöglichen. Wählt man
für seine Berechnungen obige alternative Darstellung von Hyperplanes,
so hat man durch das Skalarprodukt einen enormen Rechenaufwand 
aufgrund der hohen Dimensionalität nach Transformation seiner Daten.
Die Kernel-Functions sind im Prinzip eine spezielle Vorschrift für die
Transformation unseres Datenraumes, die es uns erlaubt, Skalarprodukte
für den höherdimensionalen Raum im ursprünglichen Raum zu berechnen.

In unserer beispielhaften Transformation oben haben wir unseren
2D-Raum in einen 4D-Raum verändert, wobei jede Dimension nun
ein Produkt aus 3 Faktoren war. Betrachten wir:
\[
x = b + \sum _{i \in SV} \alpha_i y_i \cdot \langle a(i),a\rangle^n
\]
Das hat, ausmultipliziert für \(n=3\), die selbe Struktur wie 
unsere Transformation,
also eine Linearkombination aus Produkten mit 3 Faktoren (und ein
paar Gewichten, die jedoch nur die Skalierung ändern und deswegen
im Prinzip egal sind). 
Generell ist jede Funktion eine Kernel Function, wenn sie sich als
\[
K(x,y) = \langle \phi(x), \phi(y)\rangle
\]
darstellen lässt, wobei \(\phi\) eine Abbildung in einen höherdimensionalen
Raum darstellt. Dieses Beispiel einer Kernel Function heißt Polynomial
Kernel, es gibt auch weitere, z.B. Radial, Sigmoid, etc.

\subsection{Support Vector Regression}
Bisher haben wir uns nur auf kategorische Klassifikation beschränkt.
Jetzt wollen wir uns auch der numerischen Vorhersage widmen. Diese
läuft ähnlich ab, wie die normale Lineare Regression. Wir wollen eine
Gerade finden, die den Datenbestand möglichst gut approximiert und dabei
die Fehler minimiert. Dies wollen wir jetzt mithilfe von Support Vectors tun.

Eine Darstellung mit Support Vectors ist kompakt und lässt sich mittels
einer Kernel-Function auch auf nicht-lineare Probleme anwenden.
Die Idee ist es, einen Schlauch mit Durchmesser \(2\epsilon\) um
unsere Gerade zu legen, in der Fehler \textbf{ignoriert} werden.
Die außerhalb liegenden Punkte bestimmten nun den Fehler unserer Gerade
und werden versucht zu minimieren. Außerdem wird eine möglichst flache 
Gerade gesucht. Die Flachheit einer Geraden steht im Zusammenhang mit
dem Mittelwert der Daten. Je flacher die Gerade, desto näher liegt der
vorhergesagte Wert später am Mittelwert. Die Flachheit der Geraden 
wird insbesondere von \(\epsilon\) bestimmt: Ein zu großer Wert
führt zu einem Schlauch, in den alle Datenobjekte passen; unsere Gerade
degeneriert zu einer Horizontalen und unsere Regression gibt immer
exakt den Mittelwert aus. Ein zu kleiner Wert führt dazu, dass es mehr
Objekte gibt, die bei der Fehlerberechnung eine Rolle spielen. Dies kann
leicht ein Overfitting verursachen. Generell ist der Vorteil eines
Schlauches, in dem Fehler ignoriert werden, die geringere Gefahr eines
Overfittings. Die Gerade wird durch die jenigen Punkte bestimmt, die
in die Fehlerberechnung miteinfließen, d.h. Support Vectors sind alle
Punkte, die genau auf dem Schlauch oder außerhalb lieben.
Damit lässt sich die Gerade als
\[
x = b + \sum _{i \in SV} \alpha_i \langle a(i),a\rangle
\]
darstellen.